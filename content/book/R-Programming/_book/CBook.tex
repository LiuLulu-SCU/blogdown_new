% Options for packages loaded elsewhere
\PassOptionsToPackage{unicode}{hyperref}
\PassOptionsToPackage{hyphens}{url}
%
\documentclass[
]{book}
\usepackage{amsmath,amssymb}
\usepackage{lmodern}
\usepackage{iftex}
\ifPDFTeX
  \usepackage[T1]{fontenc}
  \usepackage[utf8]{inputenc}
  \usepackage{textcomp} % provide euro and other symbols
\else % if luatex or xetex
  \usepackage{unicode-math}
  \defaultfontfeatures{Scale=MatchLowercase}
  \defaultfontfeatures[\rmfamily]{Ligatures=TeX,Scale=1}
\fi
% Use upquote if available, for straight quotes in verbatim environments
\IfFileExists{upquote.sty}{\usepackage{upquote}}{}
\IfFileExists{microtype.sty}{% use microtype if available
  \usepackage[]{microtype}
  \UseMicrotypeSet[protrusion]{basicmath} % disable protrusion for tt fonts
}{}
\makeatletter
\@ifundefined{KOMAClassName}{% if non-KOMA class
  \IfFileExists{parskip.sty}{%
    \usepackage{parskip}
  }{% else
    \setlength{\parindent}{0pt}
    \setlength{\parskip}{6pt plus 2pt minus 1pt}}
}{% if KOMA class
  \KOMAoptions{parskip=half}}
\makeatother
\usepackage{xcolor}
\IfFileExists{xurl.sty}{\usepackage{xurl}}{} % add URL line breaks if available
\IfFileExists{bookmark.sty}{\usepackage{bookmark}}{\usepackage{hyperref}}
\hypersetup{
  pdftitle={中文书模板},
  pdfauthor={刘卢路},
  hidelinks,
  pdfcreator={LaTeX via pandoc}}
\urlstyle{same} % disable monospaced font for URLs
\usepackage{color}
\usepackage{fancyvrb}
\newcommand{\VerbBar}{|}
\newcommand{\VERB}{\Verb[commandchars=\\\{\}]}
\DefineVerbatimEnvironment{Highlighting}{Verbatim}{commandchars=\\\{\}}
% Add ',fontsize=\small' for more characters per line
\usepackage{framed}
\definecolor{shadecolor}{RGB}{248,248,248}
\newenvironment{Shaded}{\begin{snugshade}}{\end{snugshade}}
\newcommand{\AlertTok}[1]{\textcolor[rgb]{0.94,0.16,0.16}{#1}}
\newcommand{\AnnotationTok}[1]{\textcolor[rgb]{0.56,0.35,0.01}{\textbf{\textit{#1}}}}
\newcommand{\AttributeTok}[1]{\textcolor[rgb]{0.77,0.63,0.00}{#1}}
\newcommand{\BaseNTok}[1]{\textcolor[rgb]{0.00,0.00,0.81}{#1}}
\newcommand{\BuiltInTok}[1]{#1}
\newcommand{\CharTok}[1]{\textcolor[rgb]{0.31,0.60,0.02}{#1}}
\newcommand{\CommentTok}[1]{\textcolor[rgb]{0.56,0.35,0.01}{\textit{#1}}}
\newcommand{\CommentVarTok}[1]{\textcolor[rgb]{0.56,0.35,0.01}{\textbf{\textit{#1}}}}
\newcommand{\ConstantTok}[1]{\textcolor[rgb]{0.00,0.00,0.00}{#1}}
\newcommand{\ControlFlowTok}[1]{\textcolor[rgb]{0.13,0.29,0.53}{\textbf{#1}}}
\newcommand{\DataTypeTok}[1]{\textcolor[rgb]{0.13,0.29,0.53}{#1}}
\newcommand{\DecValTok}[1]{\textcolor[rgb]{0.00,0.00,0.81}{#1}}
\newcommand{\DocumentationTok}[1]{\textcolor[rgb]{0.56,0.35,0.01}{\textbf{\textit{#1}}}}
\newcommand{\ErrorTok}[1]{\textcolor[rgb]{0.64,0.00,0.00}{\textbf{#1}}}
\newcommand{\ExtensionTok}[1]{#1}
\newcommand{\FloatTok}[1]{\textcolor[rgb]{0.00,0.00,0.81}{#1}}
\newcommand{\FunctionTok}[1]{\textcolor[rgb]{0.00,0.00,0.00}{#1}}
\newcommand{\ImportTok}[1]{#1}
\newcommand{\InformationTok}[1]{\textcolor[rgb]{0.56,0.35,0.01}{\textbf{\textit{#1}}}}
\newcommand{\KeywordTok}[1]{\textcolor[rgb]{0.13,0.29,0.53}{\textbf{#1}}}
\newcommand{\NormalTok}[1]{#1}
\newcommand{\OperatorTok}[1]{\textcolor[rgb]{0.81,0.36,0.00}{\textbf{#1}}}
\newcommand{\OtherTok}[1]{\textcolor[rgb]{0.56,0.35,0.01}{#1}}
\newcommand{\PreprocessorTok}[1]{\textcolor[rgb]{0.56,0.35,0.01}{\textit{#1}}}
\newcommand{\RegionMarkerTok}[1]{#1}
\newcommand{\SpecialCharTok}[1]{\textcolor[rgb]{0.00,0.00,0.00}{#1}}
\newcommand{\SpecialStringTok}[1]{\textcolor[rgb]{0.31,0.60,0.02}{#1}}
\newcommand{\StringTok}[1]{\textcolor[rgb]{0.31,0.60,0.02}{#1}}
\newcommand{\VariableTok}[1]{\textcolor[rgb]{0.00,0.00,0.00}{#1}}
\newcommand{\VerbatimStringTok}[1]{\textcolor[rgb]{0.31,0.60,0.02}{#1}}
\newcommand{\WarningTok}[1]{\textcolor[rgb]{0.56,0.35,0.01}{\textbf{\textit{#1}}}}
\usepackage{longtable,booktabs,array}
\usepackage{calc} % for calculating minipage widths
% Correct order of tables after \paragraph or \subparagraph
\usepackage{etoolbox}
\makeatletter
\patchcmd\longtable{\par}{\if@noskipsec\mbox{}\fi\par}{}{}
\makeatother
% Allow footnotes in longtable head/foot
\IfFileExists{footnotehyper.sty}{\usepackage{footnotehyper}}{\usepackage{footnote}}
\makesavenoteenv{longtable}
\usepackage{graphicx}
\makeatletter
\def\maxwidth{\ifdim\Gin@nat@width>\linewidth\linewidth\else\Gin@nat@width\fi}
\def\maxheight{\ifdim\Gin@nat@height>\textheight\textheight\else\Gin@nat@height\fi}
\makeatother
% Scale images if necessary, so that they will not overflow the page
% margins by default, and it is still possible to overwrite the defaults
% using explicit options in \includegraphics[width, height, ...]{}
\setkeys{Gin}{width=\maxwidth,height=\maxheight,keepaspectratio}
% Set default figure placement to htbp
\makeatletter
\def\fps@figure{htbp}
\makeatother
\setlength{\emergencystretch}{3em} % prevent overfull lines
\providecommand{\tightlist}{%
  \setlength{\itemsep}{0pt}\setlength{\parskip}{0pt}}
\setcounter{secnumdepth}{5}
\usepackage{ctex}

%\usepackage{xltxtra} % XeLaTeX的一些额外符号
% 设置中文字体
%\setCJKmainfont[BoldFont={黑体},ItalicFont={楷体}]{新宋体}

% 设置边距
\usepackage{geometry}
\geometry{%
  left=2.0cm, right=2.0cm, top=3.5cm, bottom=2.5cm} 

\usepackage{amsthm,mathrsfs}
\usepackage{booktabs}
\usepackage{longtable}
\makeatletter
\def\thm@space@setup{%
  \thm@preskip=8pt plus 2pt minus 4pt
  \thm@postskip=\thm@preskip
}
\makeatother
\ifLuaTeX
  \usepackage{selnolig}  % disable illegal ligatures
\fi
\usepackage[style=apa,]{biblatex}
\addbibresource{mybib.bib}

\title{中文书模板}
\author{刘卢路}
\date{2021年11月5日}

\begin{document}
\maketitle

{
\setcounter{tocdepth}{1}
\tableofcontents
}
\hypertarget{ux7b80ux4ecb}{%
\chapter*{简介}\label{ux7b80ux4ecb}}
\addcontentsline{toc}{chapter}{简介}

R软件的bookdown扩展包是R Markdown的增强版,
支持自动目录、文献索引、公式编号与引用、定理编号与引用、图表自动编号与引用

\hypertarget{usage}{%
\chapter{中文图书Bookdown模板的基本用法}\label{usage}}

\hypertarget{usage-ins}{%
\section{安装设置}\label{usage-ins}}

\hypertarget{ux57faux7840ux8bedux6cd5}{%
\chapter{基础语法}\label{ux57faux7840ux8bedux6cd5}}

\hypertarget{ux81eaux5b9aux4e49ux51fdux6570}{%
\section{自定义函数}\label{ux81eaux5b9aux4e49ux51fdux6570}}

例子:百分制数字对应五级制分数

\begin{enumerate}
\def\labelenumi{\arabic{enumi}.}
\tightlist
\item
  先梳理一般代码段
\end{enumerate}

\begin{Shaded}
\begin{Highlighting}[]
\NormalTok{score }\OtherTok{\textless{}{-}} \DecValTok{76}
\ControlFlowTok{if}\NormalTok{ (score }\SpecialCharTok{\textgreater{}=} \DecValTok{90}\NormalTok{) \{}
\NormalTok{  res }\OtherTok{\textless{}{-}} \StringTok{" 优"}
\NormalTok{\} }\ControlFlowTok{else} \ControlFlowTok{if}\NormalTok{ (score }\SpecialCharTok{\textgreater{}=} \DecValTok{80}\NormalTok{) \{}
\NormalTok{  res }\OtherTok{\textless{}{-}} \StringTok{" 良"}
\NormalTok{\} }\ControlFlowTok{else} \ControlFlowTok{if}\NormalTok{ (score }\SpecialCharTok{\textgreater{}=} \DecValTok{70}\NormalTok{) \{}
\NormalTok{  res }\OtherTok{\textless{}{-}} \StringTok{" 中"}
\NormalTok{\} }\ControlFlowTok{else} \ControlFlowTok{if}\NormalTok{ (score }\SpecialCharTok{\textgreater{}=} \DecValTok{60}\NormalTok{) \{}
\NormalTok{  res }\OtherTok{\textless{}{-}} \StringTok{" 及格"}
\NormalTok{\} }\ControlFlowTok{else}\NormalTok{ \{}
\NormalTok{  res }\OtherTok{\textless{}{-}} \StringTok{" 不及格"}
\NormalTok{\}}
\NormalTok{res}
\end{Highlighting}
\end{Shaded}

\begin{verbatim}
## [1] " 中"
\end{verbatim}

\begin{enumerate}
\def\labelenumi{\arabic{enumi}.}
\setcounter{enumi}{1}
\tightlist
\item
  一般函数
\end{enumerate}

\begin{Shaded}
\begin{Highlighting}[]
\NormalTok{Score\_Conv }\OtherTok{\textless{}{-}} \ControlFlowTok{function}\NormalTok{(score) \{}
  \ControlFlowTok{if}\NormalTok{ (score }\SpecialCharTok{\textgreater{}=} \DecValTok{90}\NormalTok{) \{}
\NormalTok{    res }\OtherTok{\textless{}{-}} \StringTok{" 优"}
\NormalTok{  \} }\ControlFlowTok{else} \ControlFlowTok{if}\NormalTok{ (score }\SpecialCharTok{\textgreater{}=} \DecValTok{80}\NormalTok{) \{}
\NormalTok{    res }\OtherTok{\textless{}{-}} \StringTok{" 良"}
\NormalTok{  \} }\ControlFlowTok{else} \ControlFlowTok{if}\NormalTok{ (score }\SpecialCharTok{\textgreater{}=} \DecValTok{70}\NormalTok{) \{}
\NormalTok{    res }\OtherTok{\textless{}{-}} \StringTok{" 中"}
\NormalTok{  \} }\ControlFlowTok{else} \ControlFlowTok{if}\NormalTok{ (score }\SpecialCharTok{\textgreater{}=} \DecValTok{60}\NormalTok{) \{}
\NormalTok{    res }\OtherTok{\textless{}{-}} \StringTok{" 及格"}
\NormalTok{  \} }\ControlFlowTok{else}\NormalTok{ \{}
\NormalTok{    res }\OtherTok{\textless{}{-}} \StringTok{" 不及格"}
\NormalTok{  \}}
\NormalTok{  res}
\NormalTok{\}}
\end{Highlighting}
\end{Shaded}

\begin{enumerate}
\def\labelenumi{\arabic{enumi}.}
\setcounter{enumi}{2}
\tightlist
\item
  向量化改进
\end{enumerate}

法一:for循环

\begin{Shaded}
\begin{Highlighting}[]
\NormalTok{Score\_Conv2 }\OtherTok{\textless{}{-}} \ControlFlowTok{function}\NormalTok{(score) \{}
\NormalTok{  n }\OtherTok{\textless{}{-}} \FunctionTok{length}\NormalTok{(score)}
\NormalTok{  res }\OtherTok{\textless{}{-}} \FunctionTok{vector}\NormalTok{(}\StringTok{"character"}\NormalTok{, n)}
  \ControlFlowTok{for}\NormalTok{ (i }\ControlFlowTok{in} \DecValTok{1}\SpecialCharTok{:}\NormalTok{n) \{}
    \ControlFlowTok{if}\NormalTok{ (score[i] }\SpecialCharTok{\textgreater{}=} \DecValTok{90}\NormalTok{) \{}
\NormalTok{      res[i] }\OtherTok{\textless{}{-}} \StringTok{" 优"}
\NormalTok{    \} }\ControlFlowTok{else} \ControlFlowTok{if}\NormalTok{ (score[i] }\SpecialCharTok{\textgreater{}=} \DecValTok{80}\NormalTok{) \{}
\NormalTok{      res[i] }\OtherTok{\textless{}{-}} \StringTok{" 良"}
\NormalTok{    \} }\ControlFlowTok{else} \ControlFlowTok{if}\NormalTok{ (score[i] }\SpecialCharTok{\textgreater{}=} \DecValTok{70}\NormalTok{) \{}
\NormalTok{      res[i] }\OtherTok{\textless{}{-}} \StringTok{" 中"}
\NormalTok{    \} }\ControlFlowTok{else} \ControlFlowTok{if}\NormalTok{ (score[i] }\SpecialCharTok{\textgreater{}=} \DecValTok{60}\NormalTok{) \{}
\NormalTok{      res[i] }\OtherTok{\textless{}{-}} \StringTok{" 及格"}
\NormalTok{    \} }\ControlFlowTok{else}\NormalTok{ \{}
\NormalTok{      res[i] }\OtherTok{\textless{}{-}} \StringTok{" 不及格"}
\NormalTok{    \}}
\NormalTok{  \}}
\NormalTok{  res}
\NormalTok{\}}

\CommentTok{\# 结果测试}
\NormalTok{score }\OtherTok{\textless{}{-}} \FunctionTok{seq}\NormalTok{(}\DecValTok{50}\NormalTok{, }\DecValTok{100}\NormalTok{, }\DecValTok{10}\NormalTok{)}
\FunctionTok{Score\_Conv2}\NormalTok{(}\AttributeTok{score =}\NormalTok{ score)}
\end{Highlighting}
\end{Shaded}

\begin{verbatim}
## [1] " 不及格" " 及格"   " 中"     " 良"     " 优"     " 优"
\end{verbatim}

法二:利用map 系列函数

\begin{Shaded}
\begin{Highlighting}[]
\NormalTok{score }\OtherTok{\textless{}{-}} \FunctionTok{c}\NormalTok{(}\DecValTok{25}\NormalTok{, }\DecValTok{67}\NormalTok{, }\DecValTok{100}\NormalTok{)}
\NormalTok{purrr}\SpecialCharTok{::}\FunctionTok{map\_chr}\NormalTok{(score, Score\_Conv)}
\end{Highlighting}
\end{Shaded}

\begin{verbatim}
## [1] " 不及格" " 及格"   " 优"
\end{verbatim}

\hypertarget{ux6570ux636eux64cdux4f5c}{%
\chapter{数据操作}\label{ux6570ux636eux64cdux4f5c}}

\hypertarget{ux53efux89c6ux5316ux4e0eux5efaux6a21}{%
\chapter{可视化与建模}\label{ux53efux89c6ux5316ux4e0eux5efaux6a21}}

\hypertarget{section}{%
\section{}\label{section}}

\printbibliography

\end{document}
